\section{Sécurité}
\author{Kévin Moreau}


\begin{frame}
	\frametitle{Objectifs}

	\begin{block}{L'entreprise Dynamease}
	 \begin{itemize}
      \item Découvrir des failles de sécurité;
      \item Proposer des solutions à ces failles;
      \item Mettre en place un certificat de sécurité.
	 \end{itemize}
	\end{block}
\end{frame}

\begin{frame}
	\frametitle{Mise en place d'un certificat HTTPS (1/3) : Étude du cahier des charges}

	\begin{block}{Cahier des charges}
	 \begin{itemize}
      \item Transférer les requêtes HTTP vers du HTTPS;
      \item Prendre un certificat Gandi;
      \item Utiliser Nginx;
      \item Encadrer un jeune décrocheur Dynamease.
	 \end{itemize}
	\end{block}
\end{frame}

\begin{frame}
	\frametitle{Mise en place d'un certificat HTTPS (2/3) : Présentation du projet de décrocheur à développeur}

	\begin{block}{Les objectifs}
	 Proposer une formation, dans une entreprise, à des jeunes ayant quitter le parcours scolaire.
	\end{block}
\end{frame}

\begin{frame}
	\frametitle{Mise en place d'un certificat HTTPS (3/3) : Mise en place de la formation}

	\begin{block}{Cahier des charges}
	 \begin{enumerate}
      \item Mise en place d'une configuration standard Nginx ;
	  \item Ajout d'une clef non certifiée;
	  \item Ajout d'une clef certifiée Gandi. 
	 \end{enumerate}
	\end{block}
\end{frame}

\begin{frame}
	\frametitle{Recherche des failles (1/2) : Méthode suivie}

	\begin{block}{Les étapes suivies}
		\begin{itemize}
			\item Lecture du code source de l’application ;
			\item Lister les éventuelles failles critiques qui pourraient être présentes ;
			\item Effectuer les manipulations afin de mettre en avant ces failles ;
			\item Déterminer les solutions à ces problèmes.
		\end{itemize}
	\end{block}
\end{frame}

\begin{frame}
	\frametitle{Recherche des failles (2/2) : Les failles trouvée}

	\begin{block}{Liste des failles}
	 \begin{itemize}
      \item Requête non vérifiée ;
	  \item Cookies non cryptés ;
	  \item Manque de sauvegarde de données .
	 \end{itemize}
	\end{block}
\end{frame}