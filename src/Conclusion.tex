Ayant déjà effectué mon stage technicien au sein de Dynamease, je n'ai eu aucun mal à m'intégrer dans l'entreprise. Ce qui eu pour effet de pouvoir commencer au plus vite le travail qui m'était confié. De plus le fait de connaître en amont les différents employés de Dynamease m'a permis d'être beaucoup plus social avec eux et n'avoir aucun mal à mettre en avant mes idées, ou de poser diverses questions.\\

Le fait de développer une application sur deux systèmes d'exploitations différents fut très intéressant. Avoir deux points de vu différents d'une même réalisation, m'a obliger de mettre un point plus important sur la conception et sur la recherche sur les deux systèmes. De plus le développement sur IOS est limité par Apple, pour des raisons de qualité des applications de la part d'Apple. Cette limite m'a obliger à être le plus inventif possible pour pouvoir contourner ces limites et pouvoir tout de même réaliser une application égalent l'application Android. De plus la mise en production décalé entre les deux applications, du fait des vérifications d'Apple, m'a également permis d'améliorer la gestion de mon temps de travail sur les deux applications, afin de pouvoir soumettre les deux applications aux utilisateurs en même temps.

Les différentes fonctionnalités à réaliser sont actuellement proposés aux client Dynamease. Bien qu'en ce qui concerne la fonctionnalité du Click2Call, elle ne reste que prototype, l'objectif est remplie car il consistait à prouver qu'il était possible de fournir un tel procédé aux clients Dynamease. D'autres fonctionnalités ont été réalisé en parallèle de celles présentée mais elle ne nécessité pas de présentation, comme par exemple la possibilité de rappeler des contacts depuis le journal d'appel.\\

Le développement d'un outil permettant l'installation de l'environnement complet de Dynamease fut très intéressant de part du fait qu'il a fallut me former sur les différents services utilisés par Dynamease. Cet apprentissage m'a permis de comprendre toutes les technologies utilisées ainsi que leur utilité.  D'une autre part le fait de réaliser des scripts intelligents permettant de réaliser une suite de commande m'a toujours passionné dans le sens où il faut réfléchir aux différents cas d'utilisation du script. Cet outil est complet et ne nécessite pas forcément d'améliorations, car il est actuellement utilisé par tout les employées de Dynamease ainsi que les consultants. Cependant un manuel d'utilisation pourrait être créé afin de facilité encore plus l'utilisation de l'outil.\\ 

L'amélioration de la sécurité Dynamease m'a fait comprendre les enjeux de cryptologie au sein d'une entreprise. La recherche de failles est un procédé très intéressant dans le sens où il faut observer toutes les parties de l'application en se forçant à avoir un œil extérieur et critique. C'est d’ailleurs durant la recherche de ces failles que j'ai pu remarquer quelques anomalies dans ce qui était réaliser, et sur les côtés non ergonomique de certaines parties de l'application. De plus devoir apporter une solution et mesurer l'état critique de la faille, permet de se rendre compte des erreurs à éviter dans le développement d'un application. Me rendre compte de certaines erreurs sur ce service m'a permis de critiquer la fiabilité de chacune des fonctionnalités que je crée. Dans le reste de mon stage je devrais mettre en place les procédures pour régler ces différentes failles.\\

Les avantages du travail en start up sont que mon travail n'était pas brider, je pouvais réaliser mes objectifs sans avoir de réelles contraintes. Les seuls contraintes qui m'étaient données été celles que je m'obligeais à suivre. Mon avis importer beaucoup dans la réalisation des fonctionnalités, je ne me sentais pas comme un stagiaire mais comme un employé à pars entière dans l'entreprise Dynamease. Mon regrat dans le travail dans ce genre d'entreprise c'est de ne  pas avoir été en contact avec beaucoup d'employés développeur ne m'a pas permis de partager leur méthode de travail. Malgré ce point, je pense que le manque de ce partage me permet de former ma propre méthode de travail en réalisant des erreurs. Ainsi je comprend l'enjeu de ma méthode et pourquoi elle est nécessaire.\\

Ce qu'il me reste à faire dans la suite de mon stage c'est de réaliser la fonctionnalité Click2Call pour les application Iphone. De plus maintenant que la liste des outils de la performance étant réalisé il serait intéressant de mettre en place ces différents outils afin qu'ils soient utilisable par Dynamease. De plus la réalisation du transfert d'appel n'a pas pu être effectuée par faute de temps, aussi bien sur les applications téléphonique que sur les serveurs Dynamease et Asterisk.

Il est cependant difficile de déterminer les améliorations possible des fonctionnalités ajoutés sachant que celle-ci ont plutôt était rapide. Ces fonctionnalités ont donc eu le temps d'être corrigé et amélioré tout au long de ce stage.\\

La résolution des failles de sécurité, bien qu’intéressante n'a pas était réaliser durant cette première partie du stage. Leur priorité de réalisation était nettement moins importante que la réalisation des nouvelles fonctionnalités promises pour la version 3.0 de Dynamease. Cette version étant mise en place, j'aurais beaucoup plus de temps pour régler ces failles avant la fin de mon stage. J'aurais dû cependant mettre plus en avant l'importance de ces failles afin de les régler au plus vite, même si cela engendrait un retard sur la sortie de la version 3.0 de Dynamease. 