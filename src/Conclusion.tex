Ayant déjà effectué mon stage technicien au sein de Dynamease, je n'ai eu aucun mal à m'intégrer dans l'entreprise. Ce qui eut pour effet de pouvoir commencer au plus vite le travail qui m'était confié. De plus le fait de connaître en amont les différents employés de Dynamease m'a permis d'être beaucoup plus social avec eux et n'avoir aucun mal à mettre en avant mes idées, ou de poser diverses questions.\\

Le fait de développer une application sur deux systèmes d'exploitation différents fut très intéressant. Avoir deux points de vu différents d'une même réalisation, m'a obligé à mettre plus d'importance sur la conception et sur la recherche de technologies sur les deux systèmes. De plus le développement sur IOS est limité par Apple, pour des raisons de qualité des applications. Cette limite m'a obligé à être le plus inventif possible pour pouvoir la contourner et pouvoir réaliser une application plus ou moins identique à l'application Android. De plus la mise en production décalée entre les deux applications, du fait des vérifications effectuées par Apple, m'a également permis d'améliorer la gestion de mon temps de travail sur les deux applications, afin de pouvoir soumettre les deux applications aux utilisateurs en même temps.

Les différentes fonctionnalités à réaliser sont actuellement proposées aux clients Dynamease. Bien qu'en ce qui concerne la fonctionnalité du Click2Call, elle ne soit encore qu'au stade de prototype. L'objectif est rempli car il consistait à prouver qu'il était possible de fournir une telle fonctionnalité aux clients Dynamease. D'autres fonctionnalités ont été réalisé en parallèle de celles présentée, comme l'ajout de la numérotation Dynamease.\\

Le développement d'un outil permettant l'installation de l'environnement complet de Dynamease fut très intéressant de part du fait qu'il a fallu me former sur les différents éléments utilisés par Dynamease. Cet apprentissage m'a permis de comprendre toutes les technologies utilisées ainsi que leur utilité. D'une autre part le fait de réaliser des scripts intelligents permettant de réaliser une suite de commandes m'a toujours intéressé dans le sens où il faut réfléchir aux différents cas d'utilisation du script. Cet outil est complet et ne nécessite pas forcément d'améliorations, car il est actuellement utilisé par tous les employées de Dynamease ainsi que les consultants. Cependant un manuel d'utilisation pourrait être créé afin de faciliter encore plus l'utilisation de l'outil.\\

L'amélioration de la sécurité Dynamease m'a fait comprendre les enjeux de la cryptologie au sein d'une entreprise. La recherche de failles est un procédé très intéressant dans le sens où il faut observer toutes les parties de l'application en se forçant à avoir un œil extérieur et critique. C'est d’ailleurs durant la recherche de ces failles que j'ai pu remarquer quelques anomalies dans ce qui était réalisé, et sur les côtés non ergonomiques de certaines parties de l'application. De plus devoir apporter une solution et mesurer l'état critique de la faille, permet de se rendre compte des erreurs à éviter dans le développement d'une application. Me rendre compte de certaines erreurs sur ce service m'a permis de critiquer la fiabilité de chacune des fonctionnalités que je crée. Durant le reste de mon stage je devrais mettre en place les procédures pour régler ces différentes failles.\\

Les avantages du travail en startup sont que mon travail n'était pas bridé, je pouvais réaliser mes objectifs sans avoir de réelles contraintes. Les seules contraintes que je suivais étaient celles que je m'obligeais à suivre. Mon avis importait beaucoup dans la réalisation des fonctionnalités, je ne me sentais pas comme un stagiaire mais comme un employé à part entière dans l'entreprise Dynamease. Mon seul regret est de ne pas avoir été en contact avec plus d'employés développeur avec lesquels j'aurais pu discuter sur les différentes méthodes de travail à employer. Malgré ce point, je pense que le manque de ce partage m'a permis de former ma propre méthode de travail en réalisant mes propres erreurs. Ainsi je comprends l'enjeu de ma méthode et pourquoi elle est nécessaire.\\

Pour la suite de mon stage je devrais mettre en place les différents outils de mesure de performance afin qu'ils soient utilisables par Dynamease. De plus la réalisation du transfert d'appel n'a pas pu être effectuée par faute de temps, aussi bien sur les applications téléphoniques que sur les serveurs Dynamease et Asterisk.

Il est cependant difficile de déterminer les améliorations possibles des fonctionnalités ajoutées sachant que celles-ci ont plutôt été rapides à réaliser. Ces fonctionnalités ont donc eu le temps d'être corrigées et améliorée tout au long de ce stage.\\

La résolution des failles de sécurité, bien qu’intéressante n'a pas été réalisée durant cette première partie du stage. Leur priorité de réalisation était nettement moins importante que la réalisation des nouvelles fonctionnalités promises pour la version 3.0 de Dynamease. Cette version étant mise en place, j'aurais beaucoup plus de temps pour régler ces failles avant la fin de mon stage. J'aurais dû cependant mettre plus en avant l'importance de ces failles afin de les régler au plus vite, même si cela engendrait un retard sur la sortie de la version 3.0 de Dynamease. 