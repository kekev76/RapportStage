\section{Réécriture des numéros de Téléphone}

La réécriture de numéros téléphoniques s'adresse aux clients Privilège. Comme il l'a été dit dans la présentation de l'entreprise, ces clients détiennent un numéro Dynamease. Ce numéro permet à l'utilisateur d'être joint sur n'importe lequel de ses appareils téléphoniques. Par contre l'inverse n'était pas mis en place, c'est à dire qu'il n'est pas possible de passer un appel avec ce numéro de téléphone.

Le but de cette fonctionnalité est de permettre à l'utilisateur Privilège, passant des appels depuis l'application téléphonique, d'avoir son numéro Dynamease affiché sur l'appareil téléphonique de la personne appelée.

\subsubsection{Étude du cahier des charges}

Le fonctionnement global de cette fonctionnalité sera une redirection.
L'utilisateur Privilège appelant un numéro, par l'application Dynamease, appellera en réalité son numéro Dynamease. Une redirection sera ensuite effectué afin de mettre en relation l'utilisateur Privilège et son contact.

Pour réaliser ce procédé on aura besoin de notre serveur téléphonique pour pouvoir redirigé les appels de l'utilisateur. Cette partie ne sera pas de mon ressort mais il sera effectué une explication de son fonctionnement. Cette explication est nécessaire à la compréhension du travail effectué par la suite.

Le serveur Dynamease sera également nécessaire, il nous servira d'intermédiaire entre le serveur téléphonique et les applications Téléphonique. Des requêtes devront donc être crée pour gérer ces communications.

Les applications téléphoniques auront pour rôle de vérifier les droits de l'utilisateur, et d'envoyer les requêtes vers le serveur avec les informations nécessaire pour faire fonctionner ce procédé.

\subsubsection{Fonctionnement du serveur téléphonique}

Le serveur téléphonique aura pour rôle de faire une redirection. Cette redirection est déjà effectif avec les appels entrant sur les numéros Dynamease. Il faut juste effectuer cette redirection également pour les appels sortant.

Pour cela une condition est ajoutée lors de l'appel entrant, si le numéro de l'appel entrant sur un numéro Dynamease appartient à l'utilisateur détenant ce numéro Dynamease, alors le serveur Téléphonique déterminera qu'il s'agit d'un cas de réécriture de numéro.

Le serveur Téléphonique doit donc posséder les informations suivantes :

\begin{itemize}
	\item Les numéros de téléphone de l'utilisateur Privilège;
	\item Le numéro de téléphone de redirection.
\end{itemize}
 

\subsubsection{Le serveur Dynamease}

Le serveur Dynamease aura pour rôle de communiquer les informations entre la partie application téléphonique et serveur téléphonique. Plusieurs requêtes doivent donc être créées. La communication entre les différents services peut-être représentée ainsi :\\

[mettre diagramme]\\

On a noté dans la partie précédente les différentes informations que devait détenir le serveur téléphonique. Les différents numéros d'un utilisateurs Dynamease sont stocké sur le serveur. Il est donc aisé de récupérer ces informations, afin de déterminer si un numéro entrant appartient à l'utilisateur.

La meilleur solution est de créer une requête permettant au serveur téléphonique de vérifier si un numéro correspond à un utilisateur donné. La réponse renvoyé sera de la forme d'un booléen. Cette requête sera effectuée à chaque appel entrant sur le serveur téléphonique.

En ce qui concerne le numéro à appelé, il faudra créée une requête qui sera appelé par les application téléphonique. Cette requête devra contenir le numéro à appeler ainsi que l'Id de l'utilisateur voulant passer l'appel. Cette requête retournera alors le numéro Dynamease, afin que les applications téléphoniques puissent procéder à l'appel.

Cette requête devra effectuée elle aussi un appel de requête vers le serveur téléphonique. ainsi le numéro à appeler resterais en mémoire, jusqu'à changement, sur le serveur téléphonique. Cela permettrait à l'utilisateur, de pouvoir rappeler son contact sans pour autant passer par l'application Dynamease.

\subsubsection{Les applications Téléphoniques}

Les applications téléphoniques doivent envoyer une requête vers le serveur Dynamease dès que l'utilisateur sélectionne l'appel d'un contact. L'application doit également fournir le numéro à appeler.

Il faut savoir que les contacts sont ranger en deux catégories, les contacts Dynamease et les contacts du téléphone. Les contacts téléphoniques sont pourvu d'un numéro de téléphone alors que les contacts Dynamease n'ont uniquement que leur identifiant Dynamease. Donc selon le contact à appeler, l'application doit soit envoyer un numéro de téléphone ou le numéro de l'identifiant Dynamease du contact.

\section{Le Click2Call}

\section{L'appel depuis l'historique d'appel}

\section{Transfert d'appel}