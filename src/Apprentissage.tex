\section{Introduction}

La réalisation des applications téléphoniques nécessitent différentes technologies et outils utilisés dans la création de ses applications. Les applications Iphone et Android nécessitant des outils différents, une partie non négligeable de mon travail a été de me former sur ceux-ci. A la suite de plusieurs recherches sur les documentations officielles j'ai pu noter les points suivant pour chacun des systèmes d'exploitation téléphonique.

Malgrès la différence de ces deux système d'exploitation, il existe quelques ressemblance, en particulier sur leur utilisation du pattern MVC (Model View Controler).Le modèle s’occupe de stocker les données, la vue les affiche à l’utilisateur, en créant les différents éléments de navigation et d'interaction. Enfin le contrôleur réalise le lien entre la vue et le modèle. Il récupère les informations de la vue, pour les stocker dans le modèle. Le modèle notifie le contrôleur pour que celui-ci mette à jour la vue.

\section{Technolgie utilisée pour Android}

\subsection{Les outils utilisés}

Pour réaliser des applications Android nous devons utiliser différents outils, que ce soit pour le développement ou pour la mise en production.

\subsubsection{Eclipse}

Eclipse est un IDE (Integrated Development Environment) permettant principalement le développement Java. Android étant réalisé en Java il est possible d'intégrer des plugin permettant le développement d'application Android avec cet IDE.

\subsubsection{Le playstore}
%eclipse
%android sdk
%playstore

\subsection{Le langage de programmation}

\subsubsection{Activity}

Une activité, au sens Android, est une classe permettant l’interaction avec l'utilisateur. Cette classe à la gestion de la création d'une vue ainsi que la gestion des actions que l'utilisateur établira sur cette vue. L'activité passe par plusieurs processus qui sont représentés sur le schéma suivant.

[mettre schéma Android activity]

On peut résumer tout ces processus en quatre états :
\begin{itemize}
	\item L'activité est au premier plan et peut être contrôlée par l'utilisateur (onResume -> onPause);
	\item L'activité est visible mais les controles de l'utilisateur ne sont plus pris en charge (onPause). Dans cet état toutes les données de l'activité sont conservées, mais en cas de manque de mémoire de la part de l'appareil téléphonique, l'activité est détruite;
	\item L'activité n'est plus visible elle est stoppée. Elle réagis de la même manière que dans un état de pause en ce qui concerne la mémoire des informations;
	\item L'activité est tuée, toutes ses informations sont effacées de la mémoire.
\end{itemize}
%%http://developer.android.com/reference/android/app/Activity.html#ActivityLifecycle

\subsubsection{Intent}

L'objet Intent permet le démarrage d'une activité ou d'un service. Un service réagis comme une activité qui n'aurai pas besoin d'interface utilisateur et qui tournerais en arrière plan. Ce démarrage peut être initié par n'importe quel composant de l'application.

\subsubsection{Manifest}
Le fichier Manifest représente les informations essentielle de l'application Android afin de créer le fichier APK nécessaire au lancement de l'application Android. Un fichier APK (Android PacKage) est un paquetage de fichiers compressés pour le système d'exploitation Android.

\section{Technologie utilisée pour IOS}

\subsection{Les outils utilisés}

\subsubsection{Xcode}
Xcode est l’environnement de développement intégré (EDI) permettant de développer les applications Apple (applications bureautiques et applications mobiles). Xcode est donc l’EDI de choix par défaut pour développer une application iOS. Xcode intègre un simulateur permettant d’émuler les périphériques (iPhone, iPad) de notre choix et d’exécuter l’application sans devoir passer par un périphérique physique bien que cela soit également possible. Pour ma part j'avais à disposition deux Iphones avec différentes version d'Ios afin de tester la compatibilité sur les IOS version 7 et 8. 

\subsubsection{Itunes Connect}
iTunes Connect est un site internet permettant de gérer les applications réalisée auprès d’Apple. Pour une entreprise, une fois enregistrée au programme de développement Apple, ce site permet de définir les membres de l’équipe de développent et leurs rôles au sein de l’équipe. Je possédais les permissions techniques qui permettaient de développer des application et de les publier sur l’Appstore. L'Appstore est la plateforme de téléchargement d'applications en ligne d'apple. ITunes Connect permet aussi de définir l’ensemble des métadonnées qui seront présentées aux utilisateurs lors de leur achat de l’application sur l’AppStore (mots-clés, descriptif, captures d’écrans, prix ..).

Ce site permet également d'avoir un retour sur les utilisateurs de l'application (commentaire, nombre de téléchargement, note obtenue sur l'application ...)


\subsection{Le langage de programmation}

\subsubsection{Delegate}

L'objet délégant garde une référence sur l'objet (le délégué) à qui, il va déléguer certaines actions. La délégation se fait par l'envoi d'un message. Le message informe le délégué. Le message est ensuite traité par le délégué. Le délégué pourra répondre à ce message en renvoyant un résultat ou en mettant à jour une vue. Le système de délégation a été utilisé afin de faire transiter des informations entre les différentes vues de l'application développée. La délégation s'effectue par la définition d'un protocole décrivant les messages envoyés entre les objets.

Pour utilisé une délégation, un objet doit hériter d'un autre objet \textit{delegate}. Cette héritage pourra nécessiter une redéfinition des méthodes voulues par le développeur. Si certaines méthodes ne sont pas redéfinies, alors une méthode par défaut définie dans l'objet \textit{delegate} sera appelé. On peut comparer se procédé à une interface Java à la différence que toutes les méthodes n'ont pas à être définies.

\subsubsection{Block}

Un \textit{block} permet de définir une fonction, qui pourra être utilisée en tant que paramètre dans une autre fonction ou méthode. Ceci peut être utile lors de l'appel de méthode asynchrone qui nécessite le traitement de données reçues ou le démarrage d'un autre processus.