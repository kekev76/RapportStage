\section{Introduction}

La réalisation des applications téléphoniques nécessite différentes technologies et outils. Les applications Iphone et Android nécessitant des outils différents, une partie non négligeable de mon travail a été de me former sur ceux-ci.

Malgré la différence de ces deux systèmes d'exploitation, il existe quelques ressemblances, en particulier sur leur utilisation du pattern MVC (Model View Controler). Le modèle s’occupe de stocker les données, la vue les affiche à l’utilisateur, en créant les différents éléments de navigation et d'interaction. Enfin le contrôleur réalise le lien entre la vue et le modèle. Il récupère les informations de la vue, pour les stocker dans le modèle. Le modèle notifie le contrôleur pour que celui-ci mette à jour la vue.

À la suite de plusieurs recherches sur les documentations officielles j'ai pu noter les points suivants pour chacun des systèmes d'exploitation téléphonique.

\section{Technolgie utilisée pour Android}

\subsection{Les outils utilisés}

Pour réaliser des applications Android nous devons utiliser différents outils, que ce soit pour le développement ou pour la mise en production.

\subsubsection{Eclipse}

Eclipse est un IDE (Integrated Development Environment) permettant principalement le développement Java. Android étant réalisé en Java il est possible d'intégrer des plugins permettant le développement d'applications Android avec cet IDE. Il est nécessaire de détenir le SDK d'Android pour effectuer ce développement. Un SDK est un kit de développement pour la réalisation d'application du type défini.

\subsubsection{Le Playstore}

Le Playstore est la plateforme officielle de Google qui permet de mettre à disposition des utilisateurs des applications téléphoniques. Cette mise en production est relativement rapide, de l'ordre de quelques heures.

\subsection{Le langage de programmation}

\subsubsection{Activity}

Une activité, au sens Android, est une classe permettant l’interaction avec l'utilisateur. Cette classe gère la création d'une vue ainsi que les actions de l'utilisateur sur cette vue. L'activité passe par plusieurs étapes qui sont représentées sur le schéma suivant.

\begin{figure}[!h]
	\centering
	\includegraphics[scale=0.6]{img/activity_lifecycle.png}
	\caption{\label{activity_lifecycle} Cycle de vie d'une activité Android}
\end{figure}

On peut résumer toutes ces étapes en quatre états :
\begin{description}
	\item[Active] L'activité est au premier plan et peut être contrôlée par l'utilisateur (onResume -> onPause);
	\item[En pause] L'activité est visible mais les actions de l'utilisateur ne sont plus prises en charge (onPause). Dans cet état toutes les données de l'activité sont conservées, mais en cas de manque de mémoire de la part de l'appareil téléphonique, l'activité est détruite;
	\item[Stoppée] L'activité n'est plus visible elle est stoppée. Elle réagit de la même manière que dans un état de pause en ce qui concerne la mémoire des informations;
	\item[Détruite] L'activité est tuée, toutes ses informations sont effacées de la mémoire.
\end{description}
%%http://developer.android.com/reference/android/app/Activity.html#ActivityLifecycle

\subsubsection{Intent}

L'objet Intent permet le démarrage d'une activité ou d'un service. Un service réagi comme une activité qui n'aurait pas besoin d'interface utilisateur et qui tournerais en arrière-plan. Ce démarrage peut être initié par n'importe quel composant de l'application.

\subsubsection{Manifest}
Le fichier Manifest représente les informations essentielles de l'application Android afin de créer le fichier APK nécessaire au lancement de l'application Android. Un fichier APK (Android PacKage) est un paquetage de fichiers compressés pour le système d'exploitation Android.

\section{Technologie utilisée pour IOS}

\subsection{Les outils utilisés}

\subsubsection{Xcode}
Xcode est l’IDE permettant de développer les applications Apple (applications bureautiques et applications mobiles). Xcode est donc l’IDE de choix par défaut pour développer une application ios. Xcode intègre un simulateur permettant d’émuler les périphériques (iphone, ipad) de notre choix et d’exécuter l’application sans devoir passer par un périphérique physique bien que cela soit également possible. Pour ma part j'avais à disposition deux iphones avec différentes versions d'Ios afin de tester la compatibilité sur les IOS 7 et 8. 

\subsubsection{Itunes Connect}
ITunes Connect est un site internet permettant de gérer les applications réalisées auprès d’Apple. Pour une entreprise, une fois enregistrée au programme de développement Apple, ce site permet de définir les membres de l’équipe de développement et leurs rôles au sein de l’équipe. Je possédais les permissions techniques qui permettent de développer des applications et de les publier sur l’Appstore. 

L'Appstore est la plateforme de téléchargement d'applications en ligne d'Apple. ITunes Connect permet également de définir l’ensemble des métadonnées qui seront présentées aux utilisateurs lors de leur achat de l’application sur l’AppStore (mots-clés, descriptif, captures d’écrans, prix ..). C'est ce site qui permet de mettre en production des applications. Cette mise en production est plus longue, de l'ordre de 5 jours. Cette attente est dû au fait que des vérifications sont effectuées sur l'application par des membres de l'entreprise Apple.

Ce site permet également d'avoir un retour sur les utilisateurs de l'application (commentaires, nombre de téléchargements, note obtenue sur l'application ...)


\subsection{Le langage de programmation}

\subsubsection{Delegate}

La délégation permet, comme son nom l'indique, de déléguer des méthodes vers un autre objet. Elle se fait par l'envoi d'un message qui informe le délégué. Le message est ensuite traité par le délégué. Le délégué pourra répondre à ce message en renvoyant un résultat ou en mettant à jour une vue. Le système de délégation a été utilisé afin de faire transiter des informations entre les différentes vues de l'application développée. La délégation s'effectue par la définition d'un protocole décrivant les messages envoyés entre les objets.

Pour définir une délégation, un objet doit implémenter un objet \textit{delegate}. Cet implémentation pourra nécessiter une redéfinition des méthodes voulues par le développeur. Si certaines méthodes ne sont pas redéfinies, alors une méthode par défaut définie dans l'objet \textit{delegate} sera appelée. On peut comparer ce procédé à une interface Java à la différence que toutes les méthodes n'ont pas à être définies.

\subsubsection{Block}

Un \textit{bloc} permet de définir une fonction, qui pourra être utilisée en tant que paramètre dans une autre méthode ou une autre fonction. Ceci peut être utile lors de l'appel de méthode asynchrone qui nécessite le traitement de données reçues ou le démarrage d'un autre processus.