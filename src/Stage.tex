Mon stage se déroulera pour l'essentiel sur la partie des applications téléphoniques de Dynamease. Il me sera également demandé de travailler sur les différentes méthodes de communication de Dynamease. Celles faisant intervenir, le serveur Dynamease avec les applications téléphoniques.

On peut séparer mon travail chez Dynamease en trois partie 

\section{Performance et amélioration}

J'aurais pour charge la mesure des performances, ainsi que les améliorations qu'il serait bon d'appliquer suite aux résultats obtenus par cette mesure. 

En ce qui concerne la recherche d'outils, je devrais tout d'abord réaliser une liste des différentes mesures que je devrais disposer pour avoir un aperçu des performances de Dynamease. De plus d'autre outils pourrait être ajouté à cette liste, des outils de vérification des installation informatique de Dynamease par exemple.

Une fois cette liste effectuée je devrais rechercher des outils réalisant les différentes tâches de cette liste, et les mettre en relation a fin de déterminer quels sont les meilleurs outils selon certains critère.

Une fois la liste des outils à utilisés créé, il faudra alors gérer leurs mise en place. Leur installation et leur configuration devront être effectué. Il me sera également demandé d'être capable de former quelqu'un à l'utilisation de ces différents outils.

Une fois la mise en place de ces outils effectuer j'aurais pour charge d'améliorer les fonctions ayant une mauvais performance. De plus certains dysfonctionnements sont déjà présent, je devrais également résoudre ces différents problèmes.

Pour le fonctionnement des différents outils de mesure je devrais également mettre en place un démarrage automatique de l'environnement de Dynamease.

\section{Ajout de nouvelles fonctionnalités}

Certaines fonctionnalités sont manquante dans les applications Dynamease. Il me sera demander de créer ces différentes fonctionnalités. Comme il l'a était dit précédemment, mon travail dans l'ajout de ces fonctionnalités se focalisera sur les application Dynamease ainsi que sur les procédé de communication entre le serveur et les applications. Il sera également possible que je modifie les différentes base de données pour répondre aux besoins des fonctionnalités.

Les différentes fonctionnalités a ajouter sont les suivantes :

\begin{enumerate}
	\item La réécriture de numéro;
	\item L'appel depuis l'historique d'appel;
	\item Le transfert d'appel;
	\item Le click2call.
\end{enumerate}

Chacune de ces fonctionnalités sera expliquer dans la partie des améliorations.

\section{Sécurité}