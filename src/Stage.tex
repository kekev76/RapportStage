\section{Mes Objectifs}

 Dans le but de pouvoir proposer une version 3.0 de Dynamease aux clients Dynamease mon stage se concentrera sur les applications téléphoniques de Dynamease. Il me sera également demandé de travailler sur les différentes méthodes de communication de Dynamease. Celles faisant intervenir, le serveur Dynamease avec les applications téléphoniques.

On peut séparer mon travail chez Dynamease en trois partie 

\subsection{Amélioration de l'existant}

J'aurais pour charge d'améliorer les fonctions ayant des dysfonctionnements ou de mauvaise performance (temps de réponse trop longue par exemple). Je devrais donc résoudre ces différents problèmes, en relation directe avec les applications téléphonique.

L'installation des services Dynamease, est une opération longue et complexe. J'aurais pour rôle de complètement modifier cette installation afin de la rendre plus simple et automatisé. Cette facilité de prise en main servira au nouveau arrivant Dynamease afin qu'il puisse avoir leur propre environnement Dynamease sur leur machine locale et également pouvoir installer les services Dynamease rapidement sur un nouveau serveur sans avoir à effectuer de grosses configurations.

Il sera également intéressant que je m’intéresse aux différents outils utilisés par Dynamease dans le cadre de la mesure de performance. Après cette étude je pourrais également proposer des pistes d'amélioration en proposant des outils d'aide à cette gestion, qui ne serait pas encore utilisé et qui pourrait faciliter le développement des applications Dynamease.


\subsection{Ajout de nouvelles fonctionnalités}

Dans l'objectif de passer à une version 3.0 du service Dynamease, certaines fonctionnalités doivent être ajoutée aux applications Dynamease. Il m'est demandé de mettre en place ces différentes fonctionnalités. Dans la continuité des améliorations, mon travail se focalisera sur les applications Dynamease ainsi que sur les procédé de communication entre le serveur et les applications. Il sera également possible que je modifie les différentes base de données pour répondre aux besoins des fonctionnalités.

Les différentes fonctionnalités a ajouter sont les suivantes :

\begin{enumerate}
	\item La réécriture de numéro;
	\item L'appel depuis l'historique d'appel;
	\item Le transfert d'appel;
	\item Le Click2Call.
\end{enumerate}

Chacune de ces fonctionnalités sera expliquée par la suite.

\subsection{Sécurité}

Durant ce stage je serais également en charge de l'amélioration de la sécurité des communications entre les différents services de Dynamease. Je devrais déceler les différents risques de sécurité et régler les éventuels problèmes que je pourrais déceler.

\subsection{divers}

Comme il l'a été dit au début, je suis en charge des applications téléphonique, donc chaque problème lié à ces applications devra être corrigé par moi même.

De plus dans le projet $"\textit{de décrocheur à développeur}"$ verra le jour durant mon stage. Je serais donc en charge d'un jeune décrocheur, je devrais l'encadrer sur des projets, en l'aidant dans ses recherches et en le formant sur certaines technologie en rapport avec son projet.

Les projets que je devrais encadrer sont liés à mes sujets de stage, en particulier la mise en place de la sécurité des communications.

\section{Mes contraintes à suivre}

Pour la réalisation de ces objectifs je dois réaliser en priorité tout ce qui est attrait aux applications téléphoniques, en ce qui concerne l'amélioration et l'ajout de fonctionnalité.

Pour cela je mettrait en place pour toutes mes réalisations une documentation complète des différentes méthodes que je réaliserais. Des tests unitaire seront mis en place pour chacune de mes méthodes afin de prouver leur fonctionnement. Des tests d'intégrations seront également mis en place pour tester l'ensemble de mon projet sur des procédure réelle de test. Si tout les tests prouvent le bon fonctionnement de mes applications je serais également en charge de leur mise en production.\\

En second lieu je dois m'occuper de la mise en place des outils de performance des services de Dynamease. Et enfin de vérification de la sécurité des services Dynamease.