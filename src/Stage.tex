\section{Mes Objectifs}

 Dans le but de pouvoir proposer une version 3.0 du service Dynamease aux clients, de nouvelles fonctionnalités doivent être réalisées. Mon stage se concentrera sur les applications téléphoniques de Dynamease. Je serais en charge des différentes méthodes de communication de Dynamease, en particulier, celles faisant intervenir, le serveur Dynamease avec les applications téléphoniques.

On peut séparer mon travail chez Dynamease en trois parties :

\subsection{Amélioration de l'existant}

L'amélioration des fonctions ayant des dysfonctionnements ou de mauvaises performances (temps de réponse trop longue par exemple) aura une place importante dans mon stage. Je devrais donc résoudre ces différents problèmes, en relation directe avec les applications téléphoniques.

L'installation des services Dynamease, est une opération longue et complexe. J'aurais pour rôle de changer cette installation afin de la rendre plus simple, automatisé et rapide. Cette facilité de prise en main servira aux nouveaux arrivants Dynamease afin qu'ils puissent avoir un environnement Dynamease sur leur machine locale. Il devra également pouvoir être installé ces services rapidement sur un nouveau serveur sans avoir à effectuer de configurations préalable manuellement.

Je m’intéresserais également aux différents outils utilisés par Dynamease dans le cadre de la mesure de performance. Après cette étude je pourrais proposer des pistes d'amélioration en proposant des outils d'aide à cette gestion, qui ne seraient pas encore utilisés et qui pourraient faciliter le développement des applications Dynamease.


\subsection{Ajout de nouvelles fonctionnalités}

Dans l'objectif de passer à une version 3.0 du service Dynamease, certaines fonctionnalités doivent être ajoutées aux applications Dynamease. Il m'est demandé de mettre en place ces différentes fonctionnalités. Dans la continuité des améliorations, mon travail se focalisera sur les applications téléphoniques ainsi que sur les procédés de communication entre le serveur et les applications.

Deux fonctionnalités majeures sont à rajouter, le Click2Call qui permet l'appel vers un utilisateur Dynamease sans échange de numéro de téléphone et la réécriture de numéro qui permet d'effectuer des appels sortant avec le  numéro Dynamease.

Ces fonctionnalités seront décrites par la suite.

\subsection{Sécurité}

Durant ce stage je serais également en charge de l'amélioration de la sécurité des communications entre les différents services de Dynamease. Je devrais déceler les différents risques de sécurité et régler les éventuels problèmes que je pourrais déceler.

\subsection{divers}

Pour le projet $"\textit{de décrocheur à développeur}"$, je serais en charge d'un jeune décrocheur, je devrais l'encadrer sur des projets, en l'aidant dans ses recherches et en le formant sur certaines technologies en rapport avec son projet.

Les projets que je devrais encadrer sont liés à mes sujets de stage, en particulier la mise en place de la sécurité des communications.

\section{Mes contraintes à suivre}

Pour la réalisation de ces objectifs je dois réaliser en priorité tout ce qui est attrait aux applications téléphoniques, l'amélioration et l'ajout de fonctionnalités.

Pour cela je mettrais en place, pour toutes mes réalisations, une documentation complète des différentes méthodes. Des tests unitaires seront mis en place pour chacune de mes méthodes afin de prouver leur fonctionnement. Des tests d'intégration seront également mis en place pour tester l'ensemble de mon projet sur des procédures réelles d'utilisation. Si tous les tests prouvent le bon fonctionnement de mes applications je devrais réaliser leur mise en production.\\

En second lieu je dois m'occuper de la mise en place des outils de performances des services de Dynamease. Et enfin de vérification de la sécurité des services Dynamease.