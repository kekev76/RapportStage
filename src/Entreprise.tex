\section{Présentation de Dynamease}

\subsection{Présentation du service Dynamease}

\subsection{Les différentes offres}

Dynamease propose à ses clients plusieurs offres. Chaque offre contient des fonctionnalité particulière. Ces offres suivent une hiérarchie, chaque offre contient les fonctionnalités des offres précédentes. La hiérarchie des offres Dynamease est la suivante :

[Diagramme hiérarchie]

Nous allons maintenant décrire les différentes offres.

\subsubsection{Basic}

L'offre Basic est une offre gratuite, elle permet à l'utilisateur d'accéder au fonction basique de l'application téléphonique et de l'application web, envoie de carte de visite Dynamease, liste de contact et pouvoir gérer manuellement sa disponibilité.

\subsubsection{Avantage}

L'offre Avantage s'adresse à des clients particulier, désirant avoir une gestion de sa disponibilité gérer par rapport à des calendrier (Calendrier Dynamease ou Google).

\subsubsection{Privilège}

L'offre Privilège permet aux utilisateur d'obtenir un numéro Dynamease. Ce numéro permet à l'utilisateur d'être joint sur n'importe lequel de ses appareils téléphoniques. Les appels dirigés vers le numéro Dynamease sont géré par le serveur Dynamease, celui-ci identifiera l'appel, défini sont importance et enfin le redirige vers l'appareil défini par l'utilisateur.

\subsubsection{Intégrale}

L'offre Intégrale est destiné pour les entreprises. Cette offre permet l'ajout d'employé qui détiendront un compte Avantage. Il est également possible d'ajouter des connecteurs. Les connecteurs permettent une meilleur identification d'appel ainsi qu'une meilleur redirection. Les connecteurs sont des données rentré par les entreprises qui donne des indications sur leurs clients ainsi que sur les employés responsable de ces clients.

\subsection{Présentation de l'environnement Dynamease}