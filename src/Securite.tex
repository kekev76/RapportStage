Pour cette partie de travail, je n'avais pas de cahier des charges à proprement parler. J'étais beaucoup plus libre dans mes choix. Il fallait que je m'organise dans le but de faire une gestion pour la sécurité de Dynamease. 

Le résultat à obtenir était de lister les éventuels failles de sécurité et de proposer des moyens pour les résoudre. Dans le but de déceler ces éventuels failles j'ai suivi les étapes suivantes :

\begin{enumerate}
	\item Observation du code de sources de l'application;
	\item Lister les éventuels failles critiques qui pouvaient être présentes;
	\item Effectuer les manipulations a fin de trouver la faille;
	\item Détecter la source de cette faille;
	\item Déterminer les solutions à ce problème.
\end{enumerate} 

En plus de trouver et résoudre ces différentes failles, je me devais également de penser à mettre en place le protocole Https pour le site web de Dynamease. Cette mise en place devra être effectué par une jeune décrocheur, je serais par contre en charge de jouer le rôle de chef de projet, en lui indiquant ce qu'il doit faire, et également jouer le rôle de formateur, en lui apprenant les différentes méthodes pour arriver à l'objectif. 

\section{Vérification de sécurité des services Dynamease}

Dans cette partie j'expliquerais, pour toutes les failles de sécurité que j'ai pu trouver, les méthodes utilisées pour trouver celles-ci.

\subsection{Requêtes non sécurisées}

\subsection{Cookies non sécurisés}

Lors de l'utilisation de l'outil de mise en place d'un environnement Dynamease, je me suis aperçu qu'en changeant les données de la base de donnée une connexion utilisateur était établie, alors que l'utilisateur connecté n'existait pas sur la base de donnée précédente.

La question qui se posait alors, était qu'est ce qui déterminait qu'un utilisateur est connecté. Le serveur Dynamease détermine la connexion grâce aux cookies.

Après une observation des cookies, il est apparu que les informations contenues dans les cookies était écris en claires. L'information qui nous intéresse est l'Id Dynamease. L'identifiant Dynamease étant unique c'est grâce à celui-ci que l'utilisateur peut être identifié.

Cette information étant en claire il suffit donc de la modifier avec un autre identifiant afin d'usurper le compte d'un autre utilisateur.
\\\\

Afin de régler cette faille il suffit d'encoder les informations contenues dans les cookies. De plus d'autres informations devraient être contenue dans les cookies afin d'être certain que l'Id utilisé représente bien l'utilisateur actuel. 

\subsection{Sauvegarde des données}

\section{Mise en place d'un certificat Https}